% !TeX root = ../TdGMates_OscarSaleta.tex

\begin{titlepage}

\begin{center}




\begin{minipage}[t]{0.4\textwidth}
\begin{center}
\includegraphics[width=0.5\textwidth]{figures/uab_logo.jpg}\\[0.15cm]
{\Large Universitat Autònoma \\de Barcelona}
\end{center}
\end{minipage}
\\[1.2cm]



\HRule\\[0.4cm]
{
\Huge Màxima versemblança i bondat d'ajust
per la distribució de Pareto generalitzada}\\[0.13cm]
\rule{\textwidth}{0.4pt}\vspace*{-\baselineskip}\vspace{3.2pt}
\HRule\\[0.7cm]
{\large Oscar Saleta Reig}\\[0.6cm]
\textsc{Grau en Matemàtiques, Facultat de Ciències}\\[0.6cm]

%\vfill
%\begin{tabular}{|p{0.25\textwidth}|p{0.55\textwidth}|}
%\hline
%\textbf{Keywords:} & \textbf{Abstract:}\\
%\footnotesize\tabitem Quantum trajectories  & \multirow{6}{*}{\footnotesize\parbox{0.55\textwidth}{Abstract goes here. Abstract goes here. Abstract goes here. Abstract goes here. Abstract goes here. Abstract goes here. Abstract goes here. Abstract goes here. Abstract goes here. Abstract goes here. Abstract goes here. Abstract goes here. Abstract goes here. Abstract goes here. Abstract goes here.}} \\
%\footnotesize\tabitem De Broglie-Bohm & \\
%\footnotesize\tabitem Quantum Kepler laws & \\
% & \\
% & \\
% & \\
%\hline
%\end{tabular}

%\vfill
%\begin{center}
%\begin{tabular}{p{0.85\textwidth}}
%\rule{0pt}{2.6ex}\footnotesize{\textbf{Abstract:} This thesis focuses in the study of quantum states that carry orbital angular momentum using tools provided by the de Broglie-Bohm quantum formulation, such as quantum trajectories. These trajectories will allow us to visualize and compute orbital angular momentum for states that present suitable symmetry. Conditions under which this method yields results that agree with classical predictions will be derived from the presentation of the pilot-wave formulation in cylindrical coordinates, and several simulations for a quantum harmonic oscillator will be performed. Finally, simulations for a $-1/r$ potential will open the discussion about the possibility of a quantum equivalent of Kepler's laws using Bohmian corrections.} \\
%\rule{0pt}{2.6ex}\footnotesize{\textbf{Keywords:} Quantum Mechanics, de Broglie-Bohm, Quantum trajectories, Orbital angular momentum, Numerical simulation}\\
%\end{tabular}
%\end{center}


\vfill
\begin{minipage}{0.7\textwidth}
\begin{flushleft}
{Treball de grau dirigit per}
\end{flushleft}
\end{minipage}
\begin{minipage}{0.2\textwidth}
\begin{flushright}
{\phantom{s}}
\end{flushright}
\end{minipage}\\[0.3cm]
\begin{minipage}{0.7\textwidth}
\begin{flushleft}
{\large Dr. Joan del Castillo (Dept. de Matemàtiques, UAB) }
\end{flushleft}
\end{minipage}
\begin{minipage}{0.2\textwidth}
\begin{flushright}
{ 02/09/2015 }
\end{flushright}
\end{minipage}






\end{center}

\end{titlepage}
