% !TeX root = ../TFM.tex

\chapter{Final remarks and future prospect}

In this Masters Thesis we developed a set of computer software that will allow us to attack not only the 3 monomial centers characterization that we discussed in the previous chapter, but also other problems which will require the computational power of PBala and the vectorised polynomial operations developed here.

The development of PBala was without doubt one of the biggest challenges faced during the months of work in this Masters Thesis. A complete interface which enables not only the author of this Thesis, but also any other researcher with access to the computing server of the Mathematics Department of the Autonomous University of Barcelona, or an equivalent, PVM-enabled, computing cluster, to easily tackle problems of a greater order of magnitude than what was available before without investing any effort in writing ad-hoc scripts or running several programs by hand to distribute the execution of their simulations.

Also, the creation of the ``vectorised'' operations for planar polynomials opens up the prospect not only of finishing the characterization of centers in three monomials but also new possibilities such as the search for the focus with the highest possible order in three monomials (or for other families of planar differential polynomial equations), which is a problem that has traditionally been faced by other means than this kind of \emph{brute force} computation (e.g. the highest order systems found up to now have been obtained through perturbating known centers, see \nptextcite{Gine2012}).

In the near future, the author of this work will continue his work in order to obtain results about the exhaustivity of the set of centers showed in Theorem \ref{th.1} in order to complete the work started here and complement the results obtained by \textcite{Gasull2016}.
